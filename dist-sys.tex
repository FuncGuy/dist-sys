\documentclass{article}
\PassOptionsToPackage{hyphens}{url}
\usepackage[utf8]{inputenc}
\usepackage[a4paper]{geometry}
\usepackage[english]{babel}
\usepackage[hyphens]{url}
\usepackage{doi}
\usepackage{hyperref}
\usepackage{amsmath}
\usepackage{amssymb}
\urlstyle{rm}

\begin{document}
\title{Distributed Systems}
\author{Martin Kleppmann}
\date{Michaelmas Term 2020/2021}
\maketitle

\section{Models of distributed systems}

Relationship between dist sys and networking: networking is how you get the bytes ``over the wire'' to another machine; dist sys is what you do with the bytes once they get there.

Why distribute?
- Inherently distributed: communication, collaboration (Facebook, google docs).
- Scale, performance, fault tolerance

It's easy if you can do it on one machine! Don't distribute unless you have to. 

Calendar app supports offline reads and writes (short video in airplane mode).

\begin{itemize}
\item synchronous, partially synchronous~\cite{Dwork:1988dr}, and asynchronous model
\item crash-stop, crash-recovery, and arbitrary (Byzantine) faults
\item network may drop, duplicate, reorder packets (and even sniff and spoof?)
\end{itemize}

Fundamental building blocks of distributed systems: replication and partitioning.

Replication example: two clients A and B, four servers. Server 1 receives only A's request,
server 2 receives A then B, server 3 receives B then A, and server 4 receives only B.
How do we ensure replicas become consistent with each other?

Use this as motivation for introducing causality and happens-before.
Show that physical timestamp ordering may  be inconsistent with causality.
Distinguish between A and B being concurrent, and A happening before B
(determining which one should overwrite the other).

ABD algorithm. Last writer wins. Linearizability.

notion of causality: taken from physics (relativity).
When a happened before b, that doesn't mean that a necessarily caused b; it just means that a \emph{might have} caused b.
For this reason, we sometimes say that the happens-before relationship encodes \emph{potential causality}.

However, when a and b happened independently (no message sent after a arrived before b, and no message sent after b arrived before a), we know that a \emph{cannot have caused} b and vice versa.

Hybrid Logical Clocks - Kulkarni et al, Logical physical clocks (OPODIS 2014) \url{https://doi.org/10.1007/978-3-319-14472-6_2}

Exercise. A relation R is a strict partial order if it is transitive ($\forall a,b,c.\; (a,b) \in R \wedge (b,c) \in R \Longrightarrow (a,c) \in R$) and irreflexive ($\nexists a.\; (a,a) \in R$). Show that the happens-before relation is a strict partial order.

Exercise. Show that for any two events $a$ and $b$, exactly one of the three following statements must be true: either $a \rightarrow b$, or $b \rightarrow a$, or $a \parallel b$.

% 1. Network communication basics: JSON, TCP, HTTP/REST, UDP. Hands-on: using things like tcpdump?
% 2. Faults and failures
% 3. Architectures: client-server, local networks (multicast), peer-to-peer (distributed hash tables, NAT traversal)
% 4. Programming models: services, RPC, message brokers, actors, map-reduce, stream processing, tuple spaces?
% 5. Replication
% 6. Consensus
% 7. Convergence
% 8. Time and clocks
% 9. Security (e.g. TLS), byzantine fault tolerance, and blockchains

% State machine replication

\bibliographystyle{plainurl}
\bibliography{references}{}
\end{document}
